\section{Current Approaches}

Many approaches that aid in the visual representation and interaction with
unstructured data exist in the digital and analog world alike. A good example
thereof is the classical mind map which allows the establishment of connections
between unstructured information and thereby aiding in the emergence of a
structure.

Further approaches include the categorizing of information, visually
represented as folders or labels, establishing a timeline and attaching
information to various points in time as well as placing information inside a
coordinate system which is an excellent choice for prioritizing information
based on multiple criteria. Approaches using the coordinate system include
the \textit{importance-urgency matrix} or the \textit{growth-share matrix}.

However, there currently exists no viable solution allowing multiple approaches
to be applied to the same information in a way that allows ... in an every-day
context by the general user.

For example, it is currently not possible to connect various information in a
mind map and placing the same information in a coordinate system in such a way
that the visual representation can be interactively switched from the mind map
to the coordinate system.

To substantiate the example, let's imagine a mind map containing various goals
of an individual to provide a basic overview. Now, it would be handy to also
place the same goals in a coordinate system, say an \textit{urgency-importance
matrix}. If time constraints are given, another useful visual representation is
the timeline where the goals can be attached to specific points in time. This
is all possible with already existing systems. However, it is currently not
possible to interactively switch between the visual representations esily. This
is only made possible if the same information is placed within multiple models
at the same time and a system allows interactive switching of the model.
