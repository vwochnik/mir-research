\section{Criteria of Acceptable Solutions}

The main criterion rendering a proposal acceptable is the remedy of the problem
outlined in the introduction of this article. Therefore, a proposed solution
must process information relationally instead of sequentially. The user must be
able to create relationships between bits of information easily and be able to
work with the information as in finding and retrieving information as well as
reorganizing existing information by changing relations without changing the
bits of information itself.

\iffalse
Relations -< verknüpfung (wort) unverständlich
\fi

Furthermore, a proposed solution should be relatively easy to use in comparison
to traditional \glspl{pims}. Therefore, the learning curve must not be too
steep and the user should become accustomed in relatively little amount of time.

\iffalse
Beispiel für diesen Use-Case
\fi
