\section{Criteria of Possible Solution}


\iffalse
\subsection{Selection of Criteria}

The selection of criteria for the comparison of traditional and relational information management systems is done in a satisfactory manner to obtain a conclusion that can be relied upon for further investigation.

The list of criteria contains subjective as well as objective criteria. It has to be said, however, that all criteria is subjective in some way since two completely different solutions to one problem are compared by a set of criteria. Even the selection of those criteria is a subjective task done by the author of this paper.

\subsection{Objective Criteria}

\begin{itemize}
\item \textbf{Use of resources.} The amount of computational and storage resources the system requires.
\item \textbf{Complexity of data structure.} The complexity of the data structure required for operation of the system.
\end{itemize}

\subsection{Subjective Criteria}

\begin{itemize}
\item \textbf{Ease of use.} The easiness of use of the individual system. This criterion can be further divided:
\begin{itemize}
\item \textbf{Adding new information.}
\item \textbf{Finding existing information.}
\item \textbf{Reorganizing existing information.}
\end{itemize}
\item \textbf{Learning curve.} The subjective learning experience of the individual. This criterion can be further divided into multiple sub criteria:
\begin{itemize}
\item \textbf{Rate of adaptation.} The speed an individual can adapt to a newly introduced system that was previously unknown to that individual.
\item \textbf{Presence of obstacles.} The presence and rate of solving of obstacles in the learning of the newly introduced system.
\end{itemize}
\item \textbf{General appeal.} How much the system appeals to the individual subjectively.
\end{itemize}

\fi
