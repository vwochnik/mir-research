\section{Introduction}

A personal information management system, henceforth \gls{pims}, is a system
directed towards the storage and organization of information exclusively
related to and relevant for the user of the system.

Conventional \gls{pims} such as digital to-do lists and notebooks allow the
recording and organization of information in a linear fashion meaning that
information can only be stored sequentially on pages or notes. Changing the
structure and order of the information is always coupled to the change of the
information itself since the pages the information is stored on have to be
changed. This may not be the most optimal way since the change itself is
requiring effort on the part of the user.

When information was primarily stored on analog media such as stone and paper,
sequential organization and storage of information was a necessity. Since a
large portion of the western population is nowadays equipped with an
internet-capable smartphone or a personal computer \cite{rainie2010internet}, a
sequential approach to information organization and storage may no longer be
the preferred solution since digital information processing is not restricted
to a sequential paradigm.

The human mind does not process information in a sequential manner as those
traditional \glspl{pims} do. \cite{Sowa:1984:CSI:4569} The mind
processes information in relation to other information which allows for a very
fast and efficient processing of information.

The objective of this article is to answer the question whether and how
conventional \gls{pims} can be improved to allow for a more optimal solution.

\iffalse
0 problem
1st problem
2nd introduction / storyline why not optimal
3rd criteria of possible solution
4rd proposition
5 conclusion / why system more optimal / research question
\fi
