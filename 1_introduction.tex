\section{Introduction}

A personal information management system, henceforth \gls{pims}, is a system
directed towards the storage and organization of information exclusively
relevant for the user of the system.

This paper deals with the question how and whether conventional \gls{pims}
can be further improved to appeal more strongly to technically inclined
people.

Conventional \gls{pims} such as digital to-do lists and notebooks allow the
recording and organization of information in a linear fashion meaning that
information can only be stored sequentially on pages or notes. Changing the
structure and order of the information is always coupled to the change of the
information itself since the pages the information is stored on have to be
changed.

The human mind, however, does not process information in a sequential manner
as those traditional \gls{pims} do. \cite{Sowa:1984:CSI:4569} The mind
processes information in relation to other information. This allows for a very
efficient processing of information.

When information was primarily stored on analog media such as stone and paper,
sequential organization and storage of information was a necessity. Since a
large portion of the western population is equipped with an internet-capable
smartphone or a personal computer \cite{rainie2010internet}, a sequential
approach to information organization and storage may no longer be the optimal
solution since digital information processing is not restricted to a sequential
paradigm.

\iffalse
0 problem
1st problem
2nd introduction / storyline why not optimal
3rd criteria of possible solution
4rd proposition
5 conclusion / why system more optimal / research question
\fi
